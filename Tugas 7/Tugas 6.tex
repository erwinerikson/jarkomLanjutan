\documentclass[conference]{IEEEtran}
\IEEEoverridecommandlockouts

\usepackage{cite}
\usepackage{amsmath,amssymb,amsfonts}
\usepackage{algorithmic}
\usepackage{graphicx}
\usepackage{textcomp}
\usepackage{xcolor}
\def\BibTeX{{\rm B\kern-.05em{\sc i\kern-.025em b}\kern-.08em
    T\kern-.1667em\lower.7ex\hbox{E}\kern-.125emX}}

\renewcommand{\figurename}{Gambar}
\renewcommand{\tablename}{Tabel}
\renewcommand{\IEEEkeywordsname}{Keywords}
\usepackage{caption}
\usepackage{pgf}

\begin{document}
    
\title{Intrusion Detection}

\author{\IEEEauthorblockN{Erwin Erikson}
\IEEEauthorblockA{\textit{Faculty of Information Technology} \\
\textit{Institut Teknologi Batam}\\
Batam, Indonesia \\
1822003@student.iteba.ac.id}
\and
\IEEEauthorblockN{Muhammad Al Imron}
\IEEEauthorblockA{\textit{Faculty of Information Technology} \\
\textit{Institut Teknologi Batam}\\
Batam, Indonesia \\
1822007@student.iteba.ac.id}
\and
\IEEEauthorblockN{Farhan Ghulam Hadi Saputra}
\IEEEauthorblockA{\textit{Faculty of Information Technology} \\
\textit{Institut Teknologi Batam}\\
Batam, Indonesia \\
1822014@student.iteba.ac.id}
}

\maketitle

\begin{abstract}
Meningkatnya perkembangan internet tidak terlepas dari serangan seperti malware infection. Intrusion Detection System (IDS) adalah masalah nonlinier, rumit dan berhubungan dengan data lalu lintas jaringan. IDS mencoba untuk mengidentifikasi dan memberi tahu aktivitas pengguna sebagai anomali normal. Untuk mendeteksi berbagai serangan jaringan dapat dilatih dengan melakukan percobaan menggunakan dataset NSL-KDD dengan beberapa algoritma yaitu Random Forest, K-Neighbors, SVM dan Ensemble Learning.
\end{abstract}

\begin{IEEEkeywords}
intrusion detection, Random Forest, K-Neighbors, SVM, Ensemble Learning, NSL-KDD dataset.
\end{IEEEkeywords}

\section{Pendahuluan}
Pengguna internet yang terus meningkat dari tahun ke tahun tidak terlepas dari serangan yang timbul dari teknologi jaringan seperti serangan \emph{malware infection}. Oleh karena itu, diperlukan keamanan dalam sistem komputer untuk mencegah dari serangan.

IDS pertama kali diperkenalkan oleh James Anderson et al. 
Sistem deteksi intrusi diklasifikasikan menjadi dua jenis yaitu berbasis host dan berbasis jaringan. Sistem deteksi intrusi 
berbasis host memeriksa data yang dimiliki sistem komputer individu, sedangkan sistem berbasis jaringan 
menganalisis data yang dipertukarkan antar komputer\cite{farnaaz2016random}.

Sistem deteksi kesalahan penggunaan menemukan tanda serangan yang dikenal dalam sumber daya yang dipantau. Sistem deteksi anomali menemukan serangan dengan mendeteksi perubahan pola pemanfaatan atau perilaku sistem\cite{hasan}.

Dalam melakukan deteksi serangan, dapat digunakan beberapa algoritma  yaitu Random Forest, K-Neighbors\cite{nurhadi2017aplikasi}, SVM dan Ensemble Learning\cite{zhang}. Dan tujuan dari penelitian ini adalah untuk membandingkan performa dari masing-masing algoritma.

\section{Penjelasan Teori}

\subsection{Random Forest}

Hutan acak (\emph{Random Forest}) adalah kumpulan pohon keputusan yang digunakan untuk meningkatkan akurasi, biasanya dilatih dengan metode "\emph{bagging}". Ide umum dari metode \emph{bagging} adalah bahwa kombinasi model pembelajaran meningkatkan hasil secara keseluruhan.

Keuntungan \emph{Random Forest} adalah sebagai berikut\cite{farnaaz2016random}:

\begin{enumerate}
\item Hutan yang dihasilkan dapat disimpan untuk referensi di masa mendatang.
\item Hutan acak mengatasi masalah penyesuaian.
\item Dalam akurasi RF dan kepentingan variabel secara otomatis dihasilkan.
\end{enumerate}

Flowchart dari proses pemodelan algoritma \emph{Random Forest} dapat dilihat pada ``Gambar. 1''.\vspace{6pt}

\begin{minipage}{\linewidth}
\centerline{\includegraphics[width=60mm]{Gambar/Gbr001.jpg}}
\captionof{figure}{Flowchart Algoritma Random Forest}
\label{fig1}
\end{minipage}

\subsection{K-Nearest Neighbors}

\emph{K-nearest neighbors} (knn) adalah algoritma yang berfungsi untuk melakukan klasifikasi suatu data berdasarkan data pembelajaran (\emph{train data sets}), yang diambil dari k tetangga terdekatnya (\emph{nearest neighbors}), dengan k merupakan banyaknya tetangga terdekat\cite{liao}. Beberapa formula yang digunakan adalah:

\begin{itemize}
\item \emph{Euclidean Distance}\\
Untuk mendefinisikan jarak antara dua titik yaitu titik pada data training (x) dan titik pada data testing (y), maka digunakan rumus \emph{Euclidean}\cite{nurhadi2017aplikasi}, yaitu:

\begin{equation*}
d(x,y)=\sqrt{\sum^{n}_{i=1} (x_i-y_i)^2}
\label{eq1}
\end{equation*}

dimana:

$d$ = jarak antara 2 titik

$x$ = data uji

$y$ = data latih

$i$ = merepresentasikan nilai atribut

$n$ = merupakan dimensi atribut.\vspace{10pt}

\item \emph{City Block Distance}\\
\emph{City Block Distance} umumnya dihitung antara 2 koordinat objek yang berpasangan. Ini adalah penjumlahan dari perbedaan absolut antara 2 koordinat. \emph{City Block Distance} 2-titik a dan b dengan dimensi k dihitung secara matematis menggunakan rumus berikut ini:

\begin{equation*}
d_{ij}=\sum^{k}_{i=1} | a_i-b_i |
\label{eq2}
\end{equation*}

\item \emph{Manhattan Distance}\\
\emph{Manhattan Distance} merupakan salah satu pengukuran yang paling banyak digunakan meliputi penggantian perbedaan kuadrat dengan menjumlahkan perbedaan \emph{absolute} dari variabel-variabel. Fungsi ini hanya akan menjumlahkan selisih nilai x dan y dari dua buah titik.
\vspace{10pt}

\item \emph{Minkwoski Distance}\\
\emph{Minkwoski Distance} adalah metrik dalam ruang vektor bernorma yang dapat dianggap sebagai generalisasi dari kedua jarak \emph{Euclidean} dan jarak \emph{Manhattan}. Jarak \emph{Minkowski} antara dua variabel X dan Y didefinisikan sebagai:

\begin{equation*}
d = (\sum^{n}_{i=1} | X_i-Y_i |^p)^{1/p}
\label{eq3}
\end{equation*}

Kasus di mana $p$ = 1 setara dengan jarak \emph{Manhattan} dan kasus di mana $p$ = 2 setara dengan jarak \emph{Euclidean}.

\end{itemize}

Flowchart dari proses pemodelan algoritma \emph{K-nearest neighbors} dapat dilihat pada ``Gambar. 2''\cite{lubis2020optimization}.\vspace{6pt}

\begin{minipage}{\linewidth}
\centerline{\includegraphics[width=55mm]{Gambar/Gbr002.jpg}}
\captionof{figure}{Flowchart Algoritma K-nearest neighbors}
\label{fig2}
\end{minipage}

\subsection{SVM}

Teori SVM berasal dari statistik dan prinsip dasar SVM adalah menemukan \emph{hyperplane} linier yang optimal dalam ruang fitur yang secara maksimal memisahkan dua kelas target\cite{hasan}.

Dalam kaitannya dengan fungsi kernel, fungsi diskriminan mengambil bentuk berikut:

\begin{equation*}
f(x) = \sum^{n}_i \alpha_ik(x,x_i)+b
\label{eq4}
\end{equation*}

Dalam pekerjaan ini, kernel \emph{Gaussian} telah digunakan
untuk membangun pengklasifikasi SVM.\\ \emph{Gaussian} kernel:

\begin{equation*}
K(x_i,x_j) = exp\left (- \frac{||x_i-x_j||^2}{2\sigma} \right )
\label{eq5}
\end{equation*}

\noindent dimana $\sigma$ adalah lebar fungsi.

Fungsi kernel dan parameternya harus dipilih untuk membangun pengklasifikasi SVM. Melatih SVM menemukan \emph{hyperplane} margin besar, yaitu menetapkan parameter $\alpha$.

Flowchart dari proses pemodelan algoritma SVM dapat dilihat pada ``Gambar. 3''\cite{inproceedings}.\vspace{6pt}

\begin{minipage}{\linewidth}
\centerline{\includegraphics[width=50mm]{Gambar/Gbr003.jpg}}
\captionof{figure}{Flowchart Algoritma SVM}
\label{fig3}
\end{minipage}

\subsection{Ensemble Learning}

Metode ini adalah menggabungkan beberapa fitur dengan pembelajaran \emph{Ensemble}.

Flowchart dari proses pemodelan algoritma \emph{Ensemble Learning} dapat dilihat pada ``Gambar. 4''\cite{zhang}.\vspace{6pt}

\begin{minipage}{\linewidth}
\centerline{\includegraphics[width=80mm]{Gambar/Gbr004.png}}
\captionof{figure}{Flowchart Algoritma Ensemble Learning}
\label{fig4}
\end{minipage}

\section{Metodologi}

Metodologi adalah tahapan yang akan dilakukan dalam penelitian agar dapat memenuhi tujuan sesuai dengan yang diharapkan.

Sistem deteksi intrusi jaringan dapat menganalisis paket yang ditangkap jaringan dan mendeteksi apakah itu akan menjadi penyusupan atau tidak.
Dilihat dari arsitekturnya, diagram sistem mencakup beberapa modul, yang ditunjukkan pada ``Gambar. 5''.

\noindent \begin{minipage}{\linewidth}
\centerline{\includegraphics[width=60mm]{Gambar/Metodologi_Diagram.png}}
\captionof{figure}{Blok Diagram Modul Pada Sistem}
\label{fig5}
\end{minipage}
\vspace{6pt}

Berdasarkan gambar di atas, terdapat beberapa modul yang memperkenalkan sistem deteksi intrusi jaringan seperti:

\noindent \textbf{Dataset}

Untuk melakukan pelatihan/pengujian digunakan dataset NSL-KDD dimana NSLKDDTrain sebagai data latih dan NSLKDDTest sebagai data uji.

\noindent \textbf{Packet}

Paket untuk dijadikan sumber data NIDS yang diambil dari dataset.

\noindent \textbf{Preprocessor}

Pada fase preprocessing, lalu lintas jaringan dikumpulkan dan diproses untuk digunakan sebagai input ke sistem.

\noindent \textbf{Feature Extractor}

Modul ini mengekstrak vektor fitur dari paket jaringan (catatan koneksi) dan mengirimkan vektor fitur ke modul pengklasifikasi.

\noindent \textbf{Classifier}

Fungsi modul ini adalah untuk menganalisis aliran jaringan dan menarik kesimpulan apakah terjadi intrusi atau tidak.

\noindent \textbf{Decision}

Modul ini akan mendeteksi intrusi yang terjadi.

\noindent \textbf{Knowledgebase}

Modul ini berfungsi untuk sampel pelatihan fase classifier. Sistem dapat bekerja secara efektif hanya jika telah dilatih dengan benar dan memadai. Sampel intrusi dapat disempurnakan di bawah partisipasi pengguna, sehingga kemampuan deteksi dapat terus ditingkatkan.

\section{Hasil dan Pembahasan}

Dari dataset yang digunakan, terbagi beberapa kategori serangan yaitu 0 = Normal, 1 = DoS, 2 = Probe, 3 = R2L, 4 = U2R.
Hasil pembagian setiap kategori serangan dari data latih dan data uji dapat dilihat pada ``Gambar. 6''.\vspace{6pt}

\begin{minipage}{\linewidth}
    \centerline{\includegraphics[width=80mm]{Gambar/Gbr011.jpg}}
    \captionof{figure}{Jumlah Data Setiap Kategori Serangan}
    \label{fig6}
    \end{minipage}
    \vspace{6pt}

Untuk menghitung precision dan recall dengan cepat dari setiap serangan dilakukan
tahap \emph{Prediction} dan \emph{Evaluation} (\emph{validation}) dengan \emph{confusion matrix}
seperti pada ``Gambar. 7''.

\begin{minipage}{\linewidth}
\centerline{\includegraphics[width=60mm]{Gambar/Gbr005.jpg}}
\captionof{figure}{Confusion Matrix}
\label{fig7}
\end{minipage}
\vspace{6pt}


Hasil \emph{Prediction} dan \emph{Evaluation} (\emph{validation})
metode \emph{Random Forest} semua fitur dapat dilihat
pada ``Gambar. 8''.

\begin{minipage}{\linewidth}
\centerline{\includegraphics[width=90mm]{Gambar/Gbr006.jpg}}
\captionof{figure}{Confusion Matrix Random Forest All Fitur}
\label{fig8}
\end{minipage}
\vspace{6pt}

Hasil \emph{Prediction} dan \emph{Evaluation} (\emph{validation})
metode \emph{Random Forest} 13 fitur dapat dilihat
pada ``Gambar. 9''.

\begin{minipage}{\linewidth}
\centerline{\includegraphics[width=90mm]{Gambar/Gbr007.jpg}}
\captionof{figure}{Confusion Matrix Random Forest 13 Fitur}
\label{fig9}
\end{minipage}
\vspace{6pt}

Hasil \emph{Prediction} dan \emph{Evaluation} (\emph{validation})
metode \emph{K-Neighbors} dapat dilihat
pada ``Gambar. 10''.

\begin{minipage}{\linewidth}
\centerline{\includegraphics[width=90mm]{Gambar/Gbr008.jpg}}
\captionof{figure}{Confusion Matrix K-Neighbors}
\label{fig10}
\end{minipage}
\vspace{6pt}

Hasil \emph{Prediction} dan \emph{Evaluation} (\emph{validation})
metode SVM dapat dilihat
pada ``Gambar. 11''.

\begin{minipage}{\linewidth}
\centerline{\includegraphics[width=90mm]{Gambar/Gbr009.jpg}}
\captionof{figure}{Confusion Matrix SVM}
\label{fig11}
\end{minipage}
\vspace{6pt}

Hasil \emph{Prediction} dan \emph{Evaluation} (\emph{validation})
metode \emph{Ensemble Learning} dapat dilihat
pada ``Gambar. 12''.

\begin{minipage}{\linewidth}
\centerline{\includegraphics[width=90mm]{Gambar/Gbr010.jpg}}
\captionof{figure}{Confusion Matrix Ensemble Learning}
\label{fig12}
\end{minipage}
\vspace{6pt}

Dengan \emph{confusion matrix} dilakukan penghitungan \emph{Accuracy}, \emph{Precision}, \emph{Recall},
dan \emph{F-measure} dari nilai masing-masing dalam matriks dengan menerapkan persamaan berikut:

\begin{equation*}
    Accuracy = \frac{TP + TN}{TP + FP + FN + TN}
    \label{eq6}
\end{equation*}

\begin{equation*}
    Precision = \frac{TP}{TP + FP}
    \label{eq7}
\end{equation*}

\begin{equation*}
    Recall = \frac{TP}{TP + FN}
    \label{eq8}
\end{equation*}

\begin{equation*}
    F-measure = 2 * \frac{precision * recall}{precision + recall}
    \label{eq9}
\end{equation*}

\noindent dimana:

\noindent $TP$ = True Positive

\noindent $TN$ = True Negative

\noindent $FP$ = False Positive

\noindent $FN$ = False Negative
\vspace{6pt}

Hasil evaluasi kinerja dari masing-masing model atau algoritma, dapat dilihat pada tabel-tabel berikut:

\noindent Pengujian menggunakan algoritma \emph{Random Forest} untuk semua fitur dapat dilihat pada ``Tabel. I''.\vspace{6pt}

\noindent \begin{minipage}{\linewidth}
\captionof{table}{Algoritma Random Forest untuk semua Fitur}
\begin{center}
\begin{tabular}{|l|l|l|l|l|}
\hline
\multicolumn{1}{|c|}{\textbf{}}&\multicolumn{1}{|c|}{\textbf{Accuracy}}&\multicolumn{1}{|c|}{\textbf{Precision}}&\multicolumn{1}{|c|}{\textbf{Recall}}&\multicolumn{1}{|c|}{\textbf{F-measure}} \\
\hline
DoS & 0.99796 & 0.99906 & 0.99651 & 0.99765\\
\hline
Probe & 0.99670 & 0.99675 & 0.99262 & 0.99469\\
\hline
R2L & 0.99775 & 0.94964 & 0.82014 & 0.87775\\
\hline
U2R & 0.98047 & 0.97293 & 0.96844 & 0.97290\\
\hline
\end{tabular}
\label{tab1}
\end{center}
\end{minipage}\\ \\

\noindent Pengujian menggunakan algoritma \emph{Random Forest} untuk 13 fitur dapat dilihat pada ``Tabel. II''.\vspace{6pt}

\noindent \begin{minipage}{\linewidth}
\captionof{table}{Algoritma Random Forest untuk 13 Fitur}
\begin{center}
\begin{tabular}{|l|l|l|l|l|}
\hline
\multicolumn{1}{|c|}{\textbf{}}&\multicolumn{1}{|c|}{\textbf{Accuracy}}&\multicolumn{1}{|c|}{\textbf{Precision}}&\multicolumn{1}{|c|}{\textbf{Recall}}&\multicolumn{1}{|c|}{\textbf{F-measure}} \\
\hline
DoS & 0.99796 & 0.99839 & 0.99651 & 0.99718\\
\hline
Probe & 0.99382 & 0.98973 & 0.98668 & 0.98758\\
\hline
R2L & 0.97856 & 0.97280 & 0.96525 & 0.96838\\
\hline
U2R & 0.99693 & 0.96256 & 0.83183 & 0.90644\\
\hline
\end{tabular}
\label{tab2}
\end{center}
\end{minipage}\\ \\

\noindent Pengujian menggunakan algoritma \emph{K-Neighbors} dapat dilihat pada ``Tabel. III''.\vspace{6pt}

\noindent \begin{minipage}{\linewidth}
\captionof{table}{Algoritma K-Neighbors}
\begin{center}
\begin{tabular}{|l|l|l|l|l|}
\hline
\multicolumn{1}{|c|}{\textbf{}}&\multicolumn{1}{|c|}{\textbf{Accuracy}}&\multicolumn{1}{|c|}{\textbf{Precision}}&\multicolumn{1}{|c|}{\textbf{Recall}}&\multicolumn{1}{|c|}{\textbf{F-measure}} \\
\hline
DoS & 0.99715 & 0.99678 & 0.99665 & 0.99672\\
\hline
Probe & 0.99077 & 0.98606 & 0.98508 & 0.98553\\
\hline
R2L & 0.96705 & 0.95265 & 0.95439 & 0.95344\\
\hline
U2R & 0.99703 & 0.93143 & 0.85073 & 0.87831\\
\hline
\end{tabular}
\label{tab3}
\end{center}
\end{minipage}\\ \\

\noindent Pengujian menggunakan algoritma SVM dapat dilihat pada ``Tabel. IV''.

\noindent \begin{minipage}{\linewidth}
\captionof{table}{Algoritma SVM}
\begin{center}
\begin{tabular}{|l|l|l|l|l|}
\hline
\multicolumn{1}{|c|}{\textbf{}}&\multicolumn{1}{|c|}{\textbf{Accuracy}}&\multicolumn{1}{|c|}{\textbf{Precision}}&\multicolumn{1}{|c|}{\textbf{Recall}}&\multicolumn{1}{|c|}{\textbf{F-measure}} \\
\hline
DoS & 0.99371 & 0.99107 & 0.99450 & 0.99278\\
\hline
Probe & 0.98450 & 0.96907 & 0.98365 & 0.97613\\
\hline
R2L & 0.96793 & 0.94854 & 0.96264 & 0.95529\\
\hline
U2R & 0.99632 & 0.91056 & 0.82909 & 0.84869\\
\hline
\end{tabular}
\label{tab4}
\end{center}
\end{minipage}\\ \\

\noindent Pengujian menggunakan algoritma \emph{Ensemble Learning} dapat dilihat pada ``Tabel. V''.\vspace{6pt}

\noindent \begin{minipage}{\linewidth}
\captionof{table}{Algoritma Ensemble Learning}
\begin{center}
\begin{tabular}{|l|l|l|l|l|}
\hline
\multicolumn{1}{|c|}{\textbf{}}&\multicolumn{1}{|c|}{\textbf{Accuracy}}&\multicolumn{1}{|c|}{\textbf{Precision}}&\multicolumn{1}{|c|}{\textbf{Recall}}&\multicolumn{1}{|c|}{\textbf{F-measure}} \\
\hline
DoS & 0.99808 & 0.99852 & 0.99718 & 0.99772\\
\hline
Probe & 0.99275 & 0.98765 & 0.98953 & 0.98841\\
\hline
R2L & 0.97158 & 0.95838 & 0.96409 & 0.96079\\
\hline
U2R & 0.99744 & 0.94270 & 0.88758 & 0.91119\\
\hline
\end{tabular}
\label{tab5}
\end{center}
\end{minipage}\\

\noindent Tingkat akurasi dari masing-masing serangan disajikan dalam bentuk diagram garis seperti pada ``Gambar. 13''.

\noindent \begin{minipage}{\linewidth}
\centerline{\includegraphics[width=78mm]{Gambar/Gbr012.jpg}}
\captionof{figure}{Tingkat Akurasi masing-masing Serangan}
\label{fig13}
\end{minipage}
\vspace{6pt}

\noindent Tingkat presisi dari masing-masing serangan disajikan dalam bentuk diagram garis seperti pada ``Gambar. 14''.\vspace{6pt}

\noindent \begin{minipage}{\linewidth}
\centerline{\includegraphics[width=78mm]{Gambar/Gbr013.jpg}}
\captionof{figure}{Tingkat Presisi masing-masing Serangan}
\label{fig14}
\end{minipage}
\vspace{6pt}

\noindent Tingkat sensitifitas dari masing-masing serangan disajikan dalam bentuk diagram garis seperti pada ``Gambar. 15''.\vspace{6pt}

\noindent \begin{minipage}{\linewidth}
\centerline{\includegraphics[width=78mm]{Gambar/Gbr014.jpg}}
\captionof{figure}{Tingkat Recall masing-masing Serangan}
\label{fig15}
\end{minipage}
\vspace{6pt}

\noindent Perbandingan rata-rata antara presisi dan sensitifitas dari masing-masing serangan disajikan dalam bentuk diagram garis seperti pada ``Gambar. 16''.\vspace{6pt}

\noindent \begin{minipage}{\linewidth}
\centerline{\includegraphics[width=78mm]{Gambar/Gbr015.jpg}}
\captionof{figure}{F-measure masing-masing Serangan}
\label{fig16}
\end{minipage}
\vspace{6pt}

\section{Kesimpulan}

Dalam penelitian ini, kami membandingkan beberapa model untuk sistem deteksi intrusi menggunakan \emph{Random Forest}, \emph{K-Neighbors}, \emph{Support Vector Machine}, dan \emph{Ensemble  Learning} dengan ketiga model diatas. Performa keempat pendekatan ini telah diamati berdasarkan \emph{accuracy}, \emph{precision}, \emph{recall}, dan \emph{f-measure} (\emph{F$_1$-score}).

Dari hasil pengujian, tingkat performa dari masing-masing algoritma dipengaruhi oleh banyaknya data sampel serta berdasarkan nilai prediksi dan nilai aktual pada \emph{Confusion Matrix} hasil klasifikasi yang dilakukan oleh sistem (model).

Dalam melakukan perbandingan performa, penelitian ini lebih memilih diprediksi diserang ternyata tidak diserang (FP) daripada diprediksi tidak diserang ternyata diserang (FN).
Kemampuan klasifikasi algoritma \emph{Ensemble Learning} menghasilkan nilai \emph{False Negative} lebih sedikit dibandingkan algoritma lainnya,
sehingga algoritma ini dinilai lebih tinggi tingkat akurasi dan ketepatan.
 

Hasil penelitian ini sangat berguna untuk penelitian masa depan dengan cara memaksimalkan tingkat kinerja serta meminimalkan nilai \emph{False Negative}.

\bibliographystyle{ieeetr}
\bibliography{pustaka}

\vspace{12pt}

\end{document}